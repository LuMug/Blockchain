\documentclass{article}

\usepackage{amsmath}
\usepackage{amssymb}
\usepackage{array}
\usepackage{parskip} % skip line
\usepackage{fullpage} % margins
\usepackage{hyperref} % hyper references and document setup
\usepackage{xcolor} % colored text
\usepackage{graphics} % images
\usepackage{listings} % format code
\usepackage{tikz} % tikz diagrams
\usepackage{pgfgantt} % gantt charts https://github.com/skafdasschaf/latex-pgfgantt
\usepackage{makecell} % cell scopes for tables
\usepackage[normalem]{ulem} % authour strikethrough
\usepackage{biblatex} % references
\usepackage{subfiles} % project structure - best loaded last in the preamble

\usetikzlibrary{ % tikz packages
	matrix,
	positioning,
	fit,
	cd % tikz-cd communitative diagrams
}

\hypersetup{
    colorlinks=true,
    linkcolor=black,
    urlcolor=blue,
    pdftitle={Blockchain},
    pdfpagemode=FullScreen
}

\addbibresource{./references.bib}

\definecolor{lightgray}{rgb}{0.9,0.9,0.9}

\lstdefinestyle{generic} {
    backgroundcolor=\color{lightgray},
}

\lstdefinestyle{sql} {
    language=SQL,
  	backgroundcolor=\color{lightgray},
  	breakatwhitespace=false,
    basicstyle=\footnotesize,
  	breaklines=true,
  	captionpos=b,
    commentstyle=\color{dkgreen},
  	deletekeywords={...},
  	escapeinside={\%*}{*)},
  	extendedchars=true,
  	keepspaces=true,
  	keywordstyle=\color{blue},
  	morekeywords={*,modify,MODIFY,...},
  	numbers=none,
  	showspaces=false,
  	showstringspaces=false, 
  	showtabs=false,
  	stepnumber=1,
}

\newcommand{\paragraphln}[1]{\paragraph{#1}\mbox{}\\}

\title{
    Blockchain \\
    \large Documentation}
    
\author{
    Paolo Bettelini, Gianni Grasso, \sout{Giacinto Di Santis} \\
    \large Scuola d'Arti e Mestieri di Trevano (SAMT)}

\date{}

\begin{document}

\maketitle

\pagebreak

\tableofcontents

\pagebreak

\section{Introduction}

\subfile{sections/introduction}

\pagebreak

\section{Blockchain}

\subfile{sections/blockchain}

\pagebreak

\section{Blockchain Implementation}

\subfile{sections/blockchain_impl}

\pagebreak

\section{Programming language}

\subfile{sections/programming_language}

\pagebreak

\section{Structure}

\subfile{sections/structure}

\pagebreak

\section{Conclusion}

\subfile{sections/conclusion}

\pagebreak

\section{References}

\nocite{*} % cite all entries


\printbibliography[type=online, heading=subbibliography, title=Sitography]

\pagebreak

%%%%%%%%%%%

\section{Assets}

Assets to use

\begin{lstlisting}[style=sql]
	CREATE TABLE IF NOT EXISTS block (
		id INT PRIMARY KEY,
		difficulty INT,
		tx_hash BINARY(32),
		nonce BINARY(32),
		miner BINARY(32),
		mined DATETIME
	);
\end{lstlisting}

\begin{lstlisting}[style=sql]
	CREATE TABLE IF NOT EXISTS tx (
		block_id INT,
		sender_pub BINARY(32),
		recipient BINARY(32),
		amount INT,
		timestamp DATETIME,
		last_tx_hash BINARY(32),
		signature BINARY(64) 
	);
\end{lstlisting}

\begin{lstlisting}[style=sql]
	CREATE TABLE IF NOT EXISTS wallet (
		address BINARY(32),
		amount INT
	)
\end{lstlisting}


\begin{ganttchart}[
	vgrid,
	]{1}{12}
	\gantttitle{2015}{4}
	\gantttitle{2016}{4}
	\gantttitle{2017}{4}
	% top nodes
	\ganttbar[name=holiday-2015-top,bar/.style={fill=none, draw=none}]{}{1}{1}
	\ganttbar[name=holiday-2016-top,bar/.style={fill=none, draw=none}]{}{5}{5}
	\ganttbar[name=holiday-2017-top,bar/.style={fill=none, draw=none}]{}{9}{9} \\
	
	% gantt chart contents
	\ganttgroup{Group 1}{1}{7} \\
	\ganttbar{Task 1}{1}{2} \\
	\ganttlinkedbar{Task 2}{3}{7} \ganttnewline
	\ganttmilestone{Milestone}{7} \ganttnewline
	\ganttbar{Final Task}{8}{12}
	
	% bottom nodes
	\ganttbar[name=holiday-2015-bottom,bar/.style={fill=none, draw=none}]{}{1}{1}
	\ganttbar[name=holiday-2016-bottom,bar/.style={fill=none, draw=none}]{}{5}{5}
	\ganttbar[name=holiday-2017-bottom,bar/.style={fill=none, draw=none}]{}{9}{9}
	
	% shading
	\begin{scope}
	\draw [opacity=0.2,line width=12] (holiday-2015-top) -- ($(holiday-2015-bottom)+(0,-11pt)$);
	\draw [opacity=0.2,line width=12] (holiday-2016-top) -- ($(holiday-2016-bottom)+(0,-11pt)$);
	\draw [opacity=0.2,line width=12] (holiday-2017-top) -- ($(holiday-2017-bottom)+(0,-11pt)$);
	\end{scope}
	
\end{ganttchart}

\end{document}