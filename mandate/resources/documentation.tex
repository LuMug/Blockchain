\documentclass{article}
\usepackage[utf8]{inputenc}
\usepackage{amsmath}
\usepackage{amssymb}
\usepackage{parskip}
\usepackage{fullpage}
\usepackage{hyperref}

\hypersetup{
    colorlinks=true,
    linkcolor=black,
    urlcolor=blue,
    pdftitle={Blockchain},
    pdfpagemode=FullScreen,
}

\title{Blockchain}
\author{Paolo Bettelini, Giacinto Di Santis, Gianni Grasso}
\date{}

\begin{document}

\maketitle
\tableofcontents
\pagebreak

\section{Block}

Each user owns a pair of private and public key.

All the transactions broadcasted to the network are grouped into blocks, which contain

\begin{itemize}
    \item Markle tree root hash
    \item Timestamp
    \item nBits (PoW)
    \item Nonce (PoW)
    \item Previous block hash
    \item Number of transactions
\end{itemize}

With each block being confirmed, the blockchain is created.

\section{Proof of Work}

Proof-of-Work (PoW) is a cryptographic proof that a party has spent
a certian amount of computational effort.

When a miner solves the puzzle the current block is archived, a new
block is generated and all the transactions in the previous block are confirmed.
The miner is then rewarded by the system.

\section{Proof of Stake}

\pagebreak

\section{Smart Contracts}

Smart contracs are programs associated with an address and run on the blockchain.
The nodes run code from the contract program at a relevant event, such as a received transation.

Users can interact with the contract via transactions. Contracts can often interact with other contracts
and some of them are Turing-complete.

\subsection{Deployment}

A smart contract is deployed by sending a transaction to the blockchain which includes the compiled program
as well as a special receiver address.

The program is added to the current block. When the block is added to the blockchain, the contract
will execute one time to set its initial state, at which point the smart contract will now be valid and running.

\end{document}

% COMPILE:
% lualatex documentation.tex; mv documentation.pdf ..