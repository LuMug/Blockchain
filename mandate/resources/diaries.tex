\documentclass{article}
\usepackage[utf8]{inputenc}
\usepackage{amsmath}
\usepackage{amssymb}
\usepackage{parskip}
\usepackage{fullpage}
\usepackage{hyperref}

\hypersetup{
    colorlinks=true,
    linkcolor=black,
    urlcolor=blue,
    pdftitle={Blockchain Diaries},
    pdfpagemode=FullScreen,
}

\title{Blockchain - Diaries}
\author{Paolo Bettelini, Giacinto Di Santis, Gianni Grasso}
\date{}

\begin{document}

\maketitle
\tableofcontents
\pagebreak

\section{Diaries}

\subsection*{2022-01-27}

\begin{itemize}
    \item Created GitHub repository
    \item Added Readme
    \item Created Gantt
    \item Initial analysis
    \item Documentation
    \item Gradle init
\end{itemize}

Today we started to take a look at the blockchain technology.
We started planning which pieces of software we will need to write in order
to make everything work. We also initialized the porject using Gradle.

\subsection*{2022-02-03}

\begin{itemize}
    \item Created node module
    \item Created seeder module
    \item Imported SQLite dependency
    \item Tested SQLite database
    \item Added LaTeX files to \texttt{.gitignore}
    \item Seeder packets
    \item Byte utils classes
\end{itemize}

Today we created the node and seeder module using gradle.
We focused on defining the protocol for the seeder application
(to allow for peer discovery). We wrote che code for the peer discovery
packets. We also tested SQLite, creating and interacting with a test database.

\subsection*{2022-02-05}

\begin{itemize}
    \item Design change
    \item Written seeder service
    \item Tested connection to seeder
\end{itemize}

This design change is related to how a user interacts with the blockchain. \\
Our initial idea was to develop a \textit{client} software alongside the \textit{node} software.
The \textit{client} would connect to a single node as an entry to the blockchain to broadcast its transactions.
The user would then be able to broadcast transactions to the blockchain using a web-based application. \\
This is a poor design choice since it has a number of problems.

\begin{itemize}
    \item \textbf{Problem 1} Dual-functionality: Each node would need to be able to handle and distinguish both a connection
    from a node and a connection from a client.
    \item \textbf{Problem 2} WebAssembly: Both the \textit{client} and \textit{node} software would share the same
    protocol implementation, written in Java. Since the application is web-based, we would need to
    compile the \textit{client} code to WebAssembly (WASM) to run on the browser.
    \item \textbf{Problem 2.1} Sockets: It would be challenging to allow for \textit{socket connections}
    from a Java-based WASM.
    \item \textbf{Problem 3} Peer discovery: The client software can't store the addresses of other nodes
    since it is a web-based application. Even by caching nodes using \textit{cookies}, clients will generally
    need to rely on \textit{seeders} more than they should.
\end{itemize}

The solution is to completly remove the client. Only nodes can broacast transactions throughout the network.
The web application will send requests to a server, which is also a node.
This server will route the user transactions through its node.
No WebAssembly is needed. Connection to the seeder will be established (possibly) only the first time the web
application server's node connects to the blockchain. Users can still host their own node and use them as an entry to the network, and
multiple users can host their own web application for the same purpose.

\subsection*{2022-02-10}

\begin{itemize}
    \item Designed peer discovery protocol
    \item Implemented peer discovery protocol
    \item Designed logo
    \item Designed GUI
    \item Blockchain analysis
    \item Fat Jar using Gradle
\end{itemize}

Today we have design the peer discovery protocol and implemented it (not tested).
We continued to study blockchain algorithms and also design the logo and the GUI for out web application.

\subsection*{2022-02-17}

\begin{itemize}
    \item Initial Web GUI design
    \item Initial PoW computation implementation
    \item Worked on peer discovery and communication protocol
    \item Database structure discussion
    \item Initial smart contract language commit
\end{itemize}

Today we mainly focused on implementing the peer discovery and communication
protocol.

\subsection*{2022-02-24}

\begin{itemize}
    \item Finished communication and peer discovery protocol
    \item Started defining and implementing the high-level blockchain protocol
    \item Continued to learn about cryptography and blockchain technology
\end{itemize}

Today we finished the base protocol and started implementing the high-level protocol.
The outline for the next session is to solve the problem with elliptic curve cryptography (ECDSA).

\end{document}

% COMPILE:
% lualatex diaries.tex; mv diaries.pdf ..