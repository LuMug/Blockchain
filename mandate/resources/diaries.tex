\documentclass{article}
\usepackage[utf8]{inputenc}
\usepackage{amsmath}
\usepackage{amssymb}
\usepackage{parskip}
\usepackage{fullpage}
\usepackage{hyperref}

\hypersetup{
    colorlinks=true,
    linkcolor=black,
    urlcolor=blue,
    pdftitle={Blockchain Diaries},
    pdfpagemode=FullScreen,
}

\title{Blockchain - Diaries}
\author{Paolo Bettelini, Giacinto Di Santis, Gianni Grasso}
\date{}

\begin{document}

\maketitle
\tableofcontents
\pagebreak

\section{Diaries}

\subsection*{2022-01-27}

\begin{itemize}
    \item Created GitHub repository
    \item Added Readme
    \item Created Gantt
    \item Initial analysis
    \item Documentation
    \item Gradle init
\end{itemize}

Today we started to take a look at the blockchain technology.
We started planning which pieces of software we will need to write in order
to make everything work. We also initialized the porject using Gradle.

\subsection*{2022-02-03}

\begin{itemize}
    \item Created node module
    \item Created seeder module
    \item Imported SQLite dependency
    \item Tested SQLite database
    \item Added LaTeX files to \texttt{.gitignore}
    \item Seeder packets
    \item Byte utils classes
\end{itemize}

Today we created the node and seeder module using gradle.
We focused on defining the protocol for the seeder application
(to allow for peer discovery). We wrote che code for the peer discovery
packets. We also tested SQLite, creating and interacting with a test database.

\end{document}

% COMPILE:
% lualatex diaries.tex; mv diaries.pdf ..